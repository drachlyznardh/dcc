
\documentclass[a4paper, 12pt, bookmarks, openany]{article}

\usepackage[latin1]{inputenc}
\usepackage[italian]{babel}
\usepackage[pdftex,bookmarks=true,pdfborder={0 0 0}]{hyperref}
\usepackage{listings}
\usepackage[usenames,dvipsnames]{color}
\usepackage{textcomp}
\usepackage{amssymb}
\usepackage{amsmath}
\usepackage{slashed}
\usepackage{algorithmic}
\usepackage{graphicx}

\hypersetup {
	pdfauthor={Ivan Simonini},
	pdftitle={Linguaggi Formali e Compilatori},
	pdfsubject={Questa dev'essere l'ultima volta che seguo il corso}
}

\begin{document}
\title{DCC - Dynamic Creme Caramel}
\author{Ivan Simonini}
\maketitle

\section{Scelta del progetto}
I puntatori sono una buona cosa per qualunque linguaggio. Non sono
estremamente complessi da realizzare. Lo heap si rende necessario per
l'utilizzo dinamico della memoria, reso possibile dai puntatori. Il
conteggio dei riferimenti \`e un approccio pi\`u elegante e pulito rispetto
alla dimenticanza permessa dal garbage colletion.

\section{Ridefinizione delle strutture}
Avendo a disposizione gli oggetti dell'OCaml, \'e comodo usarli per
semplificare e nascondere - dove possibile - il modo in cui le cose
funzionano veramente, perch\'e quello strato non \`e di diretta competenza e
sinceramente sono pi\'c contento di avere un oggetto con memoria, scelta che
riduce lo spreco di memoria (almeno credo) ed anche il numero di parametri.

\section{Realizzazione dei puntatori}
Oltre all'ovvia modifica al parser e all'albero di sintassi, realizzata come
visto in classe durante il corso, il puntatore non \'e che un
valore-locazione, ossia un indice di cella. Il descrittore di un puntatore,
oltre al tipo finale puntato (un gType, prossimamente), la profondit\`a del
puntatore stesso, ossia il numero di CARET e il massimo numero di
derefenziazioni.

Il nodo sintattico DeRef \`e una struttura che conta il numero di CARET che
precedono un identificatore. Se questo numero \`e minore o uguale alla
profondit\`a del puntatore identificato (ovviamente, soltanto se esso \`e un
puntatore), allora il metodo do\_defer\_value raggiunge la cella
corrispondente e procede a seguire il percorso indicato.

Il nodo sintattico Ref associa - direttamente nell'ambiente - la cella di
residenza di un identificatore (presente nel suo descrittore) qualora questo
non sia una costante.

\section{Realizzazione dello Heap}
Sostanzialmente analogo allo Store, lo heap \`e realizzato mediante
un'hashtable che associa locazioni a valori. Questa volta l'associazione non
avviene in un solo passo, poich\`e lo Heap ha la responsabilit\`a, come
obiettivo del progetto d'esame, di conteggiare i riferimenti alla cella.
Ogni descrittore - il tipo HEntry, Heap Entry - include un intero,
inizializzato a 0 al momento della creazione ed opportunamente manipolato
ad ogni modifica, che conta il numero di puntatori ad essa collegati.

\section{Malloc e Free}
Prendendo come modello il lunguaggio C, le primitive di richiesta di
allocazione e di deallocazione sono state chiamate Malloc e Free.

La Malloc
\`e una speciale espressione aritmentica che ha il compito di fare da parte
destra di un'assegnazione a puntatore, restituendo il valore locazione della
prima cella allocata. Essa comporta la creazione di una o pi\`u celle nello
Heap il cui conteggio iniziale \`e pari a zero. Se il valore viene assegnato
ad un puntatore adatto, questo valore diventa immediatamente 1. Le celle
pendenti a zero riferimenti sono patibili di riallocazione.

La Free
\`e un comando che si occupa distruggere una zona allocata,
INDIPENDENTEMENTE da quanti riferimenti ad essa esistano attualmente.

\section{Concatenazione (virtuale) di Store e Heap}
Grazie alla diversificazione tra StoreLoc e HeapLoc, \`e possibile
raggruppare virtualmente le due zone di memoria per ottenere un accesso alla
cella puntata tramite un puntatore.

\end{document}
